\documentclass{AUJarticle}
\usepackage[cmex10]{amsmath}
\usepackage[utf8x]{inputenc}
\usepackage[nocompress]{cite}
\usepackage{graphicx, multirow, booktabs, color, listings}
\usepackage{balance}
\usepackage[pdftex, pdfborder={0 0 0}]{hyperref}
%\usepackage[mathrm,colour,cntbysection]{czt}
\usepackage{zed-csp}
\usepackage{tikz-uml}

\graphicspath{{figures/}}

\pagestyle{empty}

\pdfinfo{ /Title (HiRTOS: A Multi-Core RTOS written in SPARK Ada)
  /Author (J. Germ\'an Rivera)
  /Keywords (Ada) }

\newcommand{\todo}{\textcolor{red}}

\hyphenation{}

\setcounter{page}{1}

\lstset{language=Ada,
        basicstyle=\small}

\begin{document}

\title{HiRTOS: A Multi-Core RTOS written in SPARK Ada}

\addauthor{J. Germ\'an Rivera}
{Austin, TX, USA}
{jgrivera67@gmail.com}

%\issuev{35}
%\issuen{1}
%\issued{March 2014}

\shortauthor{J. Germ\'an Rivera}
\shorttitle{HiRTOS: A Multi-Core RTOS written in SPARK Ada}

\thispagestyle{plain}

\maketitle

\begin{abstract}

This industrial presentation will describe the design of \emph{HiRTOS} (High-Integrity RTOS),
a simple real-time operating system kernel and separation kernel written in SPARK Ada.
HiRTOS targets safety-critical and security-sensitive embedded software applications
that run in multi-core microcontrollers.
HiRTOS was designed using the Z notation,
as a methodical way to capture correctness assumptions that can be expressed
as programming contracts in SPARK Ada. Z is a software modeling notation based
on discrete mathematics structures (such as sets, relations and functions)
and predicate logic.

Keywords: RTOS, multi-core, high-integrity, formal methods, Z, Ada, SPARK.

\end{abstract}

\section{Introduction}

Although there are several popular RTOSes for embedded applications that run on small
microcontrollers, most of them are not designed with high-integrity applications
in mind, and as such are written in C, a notoriously unsafe language. So, it would be desirable
to have an RTOS specifically designed for high-integrity applications, and written in a safer
language, like Ada or its subset SPARK Ada, even if application code is written in C/C++.
Modern versions of Ada and SPARK Ada have programming-by-contract constructs built-in
in the language, which allows the programmer to express correctness assumptions (contracts) as
part of the code. One challenge when doing programming-by-contract is to be aware of all the
correctness assumptions that can be checked in programming contracts. Describing software design
in a formal notation, such as the \href{http://en.wikipedia.org/wiki/Z_notation}{Z notation} \cite{Zrm, WayofZ},
can help identify/elicit correctness assumptions in a more methodical way than just thinking
of them as we write code.

Z is a software modeling notation based on discrete mathematics structures (such as sets,
relations and functions) and predicate logic. With Z, data structures can be specified in
terms of mathematical structures and their state invariants can be specified using mathematical
predicates. Specifying (designing) data structures at a higher-level of abstraction using
discrete-math structures can lead to simpler and more elegant code. The pre-conditions and
post-conditions of the operations that manipulate the data structures can also be specified
using predicates. Using Z for this purpose encourages a rigorous and methodical thought process
to elicit correctness properties, in a systematic way.
The \emph{HiRTOS} Z model of HiRTOS was checked with the \verb'fuzz' tool~\cite{Fuzz}, a
Z type-checker, that catches Z type mismatches in predicates.

The code of \emph{HiRTOS} is written in SPARK Ada \cite{SparkAda}, a high integrity
subset of the Ada programming language. HiRTOS data types were modeled in Z at a level of
abstraction that can be mapped directly to corresponding data types in SPARK Ada.

\section{HiRTOS Overview}

\subsection{Major Design Decisions}

\begin{itemize}
\item For API simplicity, inspired by the thread synchronization primitives of the C11 standard
library \cite{libcThreads}, mutexes and condition variables are the only synchronization
primitives in \emph{HiRTOS}.
Other synchronization primitives such as semaphores, event flags and message queues can be
implemented on top of mutexes and condition variables.

\item Unlike C11 mutexes, \emph{HiRTOS} mutexes can change the priority of the thread owning the
mutex. \emph{HiRTOS} mutexes support both priority inheritance and priority ceiling \cite{prioCeiling}.

\item Unlike C11 condition variables, \emph{HiRTOS} condition variables
can also be waited on while having interrupts disabled, not just while holding a mutex.

\item \emph{HiRTOS} atomic levels can be used to disable the thread scheduler or to disable interrupts
at and below a given priority or to disable all interrupts.

\item In a multi-core platform, there is one \emph{HiRTOS} instance per CPU Core. Each \emph{HiRTOS} instance is
independent of each other. No resources are shared between \emph{HiRTOS} instances. No communication between
CPU cores is supported by \emph{HiRTOS}, so that the \emph{HiRTOS} API can stay the same for both single-core
and multi-core platforms. Inter-core communication would need to be provided outside of \emph{HiRTOS},
using doorbell interrupts and mailboxes or shared memory, for example.

\item
Threads are bound to the CPU core in which they were created, for the lifetime of the thread. That is,
no thread migration between CPU cores is supported.

\item
All RTOS objects such as threads, mutexes and condition variables are allocated internally
by \emph{HiRTOS} from statically allocated internal object pools.
These object pools are just RTOS-private global arrays of the corresponding RTOS object types,
sized at compile time via configuration parameters, whose values are application-specific.
RTOS object handles provided to application code are just indices into these internal object arrays.
No actual RTOS object pointers exposed to application code.
No dynamic allocation/deallocation of RTOS objects is supported and no static allocation of RTOS
objects in memory owned by application code is supported either.

\item
All application threads run in unprivileged mode. For each thread, the only writable memory,
by default, is its own stack and global variables. Stacks of other threads are not accessible.
MMIO space is only accessible to privileged code, by default. Application driver code, other
than ISRs, must request access (read-only or read-write permission) to \emph{HiRTOS} via a
system call.

\item Interrupt service routines (ISRs) are seen as hardware-scheduled threads that have higher
priority than all software-scheduled threads. They can only be preempted by higher-priority ISRs.
They cannot block waiting on mutexes or condition variables.
\end{itemize}

\subsection{Separation Kernel Major Design Decisions}

Besides being a fully functional RTOS, HiRTOS can be used as a separation kernel \cite{SepK}.

\begin{itemize}

\item In a multi-core platform, there is one separation kernel instance per CPU Core. Each instance is
independent of each other. No resources are shared between separation kernel instances. No communication between
CPU cores is supported, so that the separation kernel API can stay the same for both single-core
and multi-core platforms. Inter-core communication would need to be provided outside of HiRTOS,
using doorbell interrupts and mailboxes or shared memory, for example.

\item Each separation-kernel instance consists of one or more partitions. A partition is a spatial and temporal
separation/isolation unit on which a bare-metal or RTOS-based firmware binary runs. Each partition
consists of one more disjoint address ranges covering portions of RAM and MMIO space that only that
partition can access. Also, each partition has its own interrupt vector table, its own set of physical
interrupts and its own global machine state. So, the firmware hosted in each partition has the illusion
that it owns an entire physical machine, with is own set of of physical peripherals, dedicated memory
and CPU core. The CPU core is time-sliced among the partitions running on the same separation kernel instance.

\item
Partitions are bound to the CPU core in which they were created. That is, no partition migration between
CPU cores is supported.

\item
Partitions are created at boot time before starting the partition scheduler on the corresponding CPU core.
Partitions cannot be destroyed or terminated.

\item
The separation kernel code itself runs in hypervisor privilege mode. All partitions run at a privilege lower
than hypervisor mode. Partitions can communicate with the separation kernel via hypervisor calls and via
traps to hypervisor mode triggered from special machine instructions such as $WFI$. The separation kernel
can communicate with partitions, by forwarding interrupts targeted to the corresponding partition.

\end{itemize}


\subsection{HiRTOS Code Architecture}

To have wider adoption of an RTOS written in bare-metal Ada, providing a C/C++ programming interface
is a must. Indeed, multiple interfaces or ``skins'' can be provided to mimic widely popular
RTOSes such as FreeRTOS \cite{freeRTOS} and RTOS interfaces such as the CMSIS RTOS2 API \cite{cmsisRTOS}.
As shown on figure \ref{HiRTOSAchitecture1}, HiRTOS has a C/C++ interface layer that porvides
a FreeRTOS skin and and a CMSIS RTOS2 skin. Both skins are implemented on top of a native C skin.
The native C skin is just a thin C wrapper that consists of a C header file containing the C
functions prototypes of the corresponding Ada subprograms of the SPARK Ada native interface of HiRTOS.

In addition to the C/C++ interface, HiRTOS should also provide an Ada runtime library (RTS) skin,
as shown on figure \ref{HiRTOSAchitecture2}, so that baremetal Ada applications that use Ada tasking
features can run on top of HiRTOS. This can be especially useful, given the limited number of
microcontroller platforms for which there is a bare-metal Ada runtime library available with the
GNAT Ada compiler. an all platforms where is avaiable now or in the future.

HiRTOS has been architected to be easily portable to any multi-core microcontroller or bare metal
platform for which a GNAT Ada cross compiler is available. All platform-dependent code is isolated in the
HiRTOS porting layer, which provides platform-independent interfaces to the rest of the HiRTOS code.
To avoid any depdendency on a platform-specific bare-metal Ada runtime library, provided by the
compiler, HiRTOS sits on top of a platform-independent portable minimal Ada runtime library.

\begin{figure}
   \begin{center}
      \scalebox{0.58}{
         \begin{tikzpicture}
            \umlemptypackage[x=0, y=0]{Bare Metal C/C++ Applications}

            \begin{umlpackage}[x=0, y=-4]{C Interface Skins Library}
               \umlemptypackage[x=-3]{CMSIS RTOS2 Skin}
               \umlemptypackage[x=3]{FreeRTOS Skin}
               \umlemptypackage[x=6, y=-3]{C Interface Skin}
            \end{umlpackage}

            \begin{umlpackage}[x=0, y=-2]{HiRTOS Library Crate}
               \umlemptypackage[x=0, y=-10]{HiRTOS}
               \umlemptypackage[x=-4, y=-12]{HiRTOS Porting Layer}
            \end{umlpackage}

            \umlemptypackage[x=2, y=-18]{Portable Minimal Ada RTS Library Crate}

            \umlimport[geometry=|-]{CMSIS RTOS2 Skin}{C Interface Skin}
            \umlimport[geometry=|-]{FreeRTOS Skin}{C Interface Skin}
            \umlimport[geometry=|-]{C Interface Skin}{HiRTOS}
            \umlimport[geometry=|-]{HiRTOS}{HiRTOS Porting Layer}
            \umlimport[]{HiRTOS}{Portable Minimal Ada RTS Library Crate}
            \umlimport[geometry=|-]{HiRTOS Porting Layer}{Portable Minimal Ada RTS Library Crate}
            \umlimport[]{Bare Metal C/C++ Applications}{CMSIS RTOS2 Skin}
            \umlimport[]{Bare Metal C/C++ Applications}{FreeRTOS Skin}
            \umlimport[geometry=-|]{Bare Metal C/C++ Applications}{C Interface Skin}
         \end{tikzpicture}
      }
   \end{center}
   \caption{HiRTOS Code Architecture for C/C++ Applications}
   \label{HiRTOSAchitecture1}
\end{figure}

\begin{figure}
   \begin{center}
      \scalebox{0.58}{
         \begin{tikzpicture}
            \umlemptypackage[x=0, y=0]{Bare Metal Ada Applications}

            \umlemptypackage[x=0, y=-4]{Ada RTS Tasking Skin Library Crate}

            \begin{umlpackage}[x=0, y=-8]{HiRTOS Library Crate}
               \umlemptypackage[x=0, y=0]{HiRTOS}
               \umlemptypackage[x=-4, y=-1]{HiRTOS Config Parameters}
               \umlemptypackage[x=-4, y=-4]{HiRTOS Porting Layer}
            \end{umlpackage}

            \umlemptypackage[x=4, y=-16]{Portable Minimal Ada RTS Library Crate}

            \umlimport[]{Ada RTS Tasking Skin Library Crate}{HiRTOS}
            \umlimport[geometry=|-]{HiRTOS}{HiRTOS Porting Layer}
            \umlimport[]{HiRTOS}{Portable Minimal Ada RTS Library Crate}
            \umlimport[geometry=|-]{HiRTOS Porting Layer}{Portable Minimal Ada RTS Library Crate}
            \umlimport[]{Bare Metal Ada Applications}{Ada RTS Tasking Skin Library Crate}
            \umlimport[geometry=-|]{Bare Metal Ada Applications}{Portable Minimal Ada RTS Library Crate}
            \umlimport[geometry=|-]{HiRTOS}{HiRTOS Config Parameters}
         \end{tikzpicture}
      }
   \end{center}
   \caption{HiRTOS Code Architecture for Ada Applications}
   \label{HiRTOSAchitecture2}
\end{figure}

Figure \ref{HiRTOSAchitecture3} shows the major code components of HiRTOS. The HiRTOS code base is
structured in three conceptual layers. The \emph{HiRTOS API} layer, the \emph{HiRTOS internals layer}
and the \emph{HiRTOS porting layer}.

The \emph{HiRTOS API} layer contains the HiRTOS public interface components.
The \verb'HiRTOS_Interrupt_Handling' Ada package contains the services to be invoked from top-level
interrupt handlers to notify HiRTOS of entering an exiting interrupt context.
\verb'HiRTOS_Memory_Protection' contains the services to protect ranges of memory and MMIO space.
\verb'HiRTOS_Thread' contains the services to create and manage threads.

The \emph{HiRTOS internals layer} contains HiRTOS-private components that are
hardware-independent.

The \emph{HiRTOS porting layer} contains hardware-dependent components that
provide hardware-independent interfaces to upper HiRTOS layers.

\begin{figure}
   \begin{center}
      \scalebox{0.58}{
         \begin{tikzpicture}
            \begin{umlpackage}{HiRTOS}
               \begin{umlpackage}{HiRTOS API}
                  \umlbasiccomponent{HiRTOS}
                  \umlbasiccomponent[name=HiRTOSinterruptHandling, x=-3, y=-2.5]{HiRTOS.Interrupt\_Handling}
                  \umlbasiccomponent[name=HiRTOSmemoryProtection, x=2, y=-2.5]{HiRTOS.Memory\_Protection}
                  \umlbasiccomponent[name=HiRTOSthread, x=-3, y=-5]{HiRTOS.Thread}
                  \umlbasiccomponent[name=HiRTOStimer, x=2, y=-5]{HiRTOS.Timer}
                  \umlbasiccomponent[name=HiRTOScondvar, x=-3, y=-7.5]{HiRTOS.Condvar}
                  \umlbasiccomponent[name=HiRTOSmutex, x=2, y=-7.5]{HiRTOS.Mutex}
               \end{umlpackage}
               \begin{umlpackage}{HiRTOS internals}
                  \umlbasiccomponent[name=RTOSprivate, x=0, y=-11]{RTOS\_Private}
                  \umlbasiccomponent[name=HiRTOSinterruptHandlingPrivate, x=-3, y=-13.5]{HiRTOS.Interrupt\_Handling\_Private}
                  \umlbasiccomponent[name=HiRTOSmemoryProtectionPrivate, x=3, y=-13.5]{HiRTOS.Memory\_Protection\_Private}
                  \umlbasiccomponent[name=HiRTOSthreadPrivate, x=-3, y=-16]{HiRTOS.Thread\_Private}
                  \umlbasiccomponent[name=HiRTOStimerPrivate, x=2, y=-16]{HiRTOS.TimerPrivate}
                  \umlbasiccomponent[name=HiRTOScondvarPrivate, x=-3, y=-18.5]{HiRTOS.CondvarPrivate}
                  \umlbasiccomponent[name=HiRTOSmutexPrivate, x=2, y=-18.5]{HiRTOS.MutexPrivate}
               \end{umlpackage}
               \umlbasiccomponent[name=GenericLinkedList, x=2, y=-21.5]{Generic\_Linked\_List}
               \umlbasiccomponent[name=GenericExecutionStack, x=-3, y=-21.5]{Generic\_Execution\_Stack}
            \end{umlpackage}

            \begin{umlpackage}{HiRTOS Porting Layer}
               \begin{umlpackage}{Cpu Architecture Specific}
                  \umlbasiccomponent[name=HiRTOScpuArchParameters, x=-3, y=-26]{HiRTOS\_Cpu\_Arch\_Parameters}
                  \umlbasiccomponent[name=HiRTOScpuArchInterface, x=-3, y=-28.5]{HiRTOS\_Cpu\_Interface}
                  \umlbasiccomponent[name=HiRTOScpuStartupInterface, x=-3, y=-31]{HiRTOS\_Cpu\_Startup\_Interface}
                  \umlbasiccomponent[name=HiRTOScpuMultiCoreInterface, x=-3, y=-33.5]{HiRTOS\_Cpu\_Multi\_Core\_Interface}
               \end{umlpackage}
               \begin{umlpackage}{Platform Specific}
                  \umlbasiccomponent[name=HiRTOSPlatformParameters, x=3, y=-26]{HiRTOS\_Platform\_Parameters}
                  \umlbasiccomponent[name=HiRTOScpuArchInterfaceInterrupts, x=3, y=-28.5]{HiRTOS\_Cpu\_Interface.Interrupts}
                  \umlbasiccomponent[name=HiRTOSlowLevelDebugInterface, x=3, y=-31]{HiRTOS\_Low\_Level\_Debug\_Interface}
               \end{umlpackage}
            \end{umlpackage}

            \umlimport[]{HiRTOS API}{HiRTOS internals}
            \umlimport[]{HiRTOS internals}{GenericLinkedList}
            \umlimport[]{HiRTOS}{HiRTOS Porting Layer}
         \end{tikzpicture}
      }
   \end{center}
   \caption{HiRTOS Code Components}
   \label{HiRTOSAchitecture3}
\end{figure}

%\bibliographystyle{ieeetr}
%\bibliography{bibliography}
%\balance

\begin{thebibliography}{9}

\bibitem{Zrm}
Mike Spivey, ``The Z Reference Manual'', second edition, Prentice-Hall, 1992 \\
\url{http://spivey.oriel.ox.ac.uk/~mike/zrm/zrm.pdf}

\bibitem{WayofZ}
Jonathan Jacky, ``The Way of Z'', Cambridge Press, 1997 \\
\url{http://staff.washington.edu/jon/z-book/index.html}

\bibitem{Fuzz}
Mike Spivey, ``The Fuzz checker'' \\
\url{http://spivey.oriel.ox.ac.uk/mike/fuzz}

\bibitem{SparkAda}
John W. McCormick, Peter C. Chapin,``Building High Integrity Applications with SPARK'', Cambridge University Press, 2015 \\
\url{https://www.amazon.com/Building-High-Integrity-Applications-SPARK/dp/1107040736}

\bibitem{gnatprove}
AdaCore,``Formal Verification with GNATprove'' \\
\url{https://docs.adacore.com/spark2014-docs/html/ug/en/gnatprove.html}

\bibitem{libcThreads}
ISO, ``N2731: Working draft of the C23 standard, section 7.26'', October 2021 \\
\url{http://www.open-std.org/jtc1/sc22/wg14/www/docs/n2596.pdf#page=345&zoom=100,102,113}

\bibitem{prioCeiling}
Lui Sha et al, ``Priority Inheritance Protocols: An Approach to Real-Time Synchronization'', IEEE Transactions on Computers, September 1990 \\
\url{https://www.csie.ntu.edu.tw/~r95093/papers/Priority%20Inheritance%20Protocols%20An%20Approach%20to%20Real-Time%20Synchronization.pdf}

\bibitem{SepK}
John Rushby, ``Design and Verification of Secure Systems'', ACM SIGOPS Operating Systems Review, 1981 \\
\url{https://www.csl.sri.com/users/rushby/papers/sosp81.pdf}

\bibitem{freeRTOS}
FreeRTOS
\url{https://www.freertos.org/}

\bibitem{cmsisRTOS}
CMSIS-RTOS API v2 (CMSIS-RTOS2)
\url{https://www.keil.com/pack/doc/CMSIS/RTOS2/html/group__CMSIS__RTOS.html}

\end{thebibliography}

\end{document}
